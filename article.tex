\documentclass[sigplan,10pt,nonacm=true]{acmart}
\settopmatter{printfolios=false,printccs=false,printacmref=false}
\usepackage[underline=true,rounded corners=false]{pgf-umlsd} % for UML sequence diagrams
\usepackage[square]{syntax} % for BNF grammar
  \setlength{\grammarparsep}{1pt}
  \setlength{\grammarindent}{2em}
\usepackage{array} % complex structure of TABULAR-s
\usepackage{paralist} % for inline itemized lists env{inparaenum}
\usepackage{natbib} % for customization of bibliography headers
\usepackage{ffcode}
\usepackage{tikz}
  \usetikzlibrary{shapes,arrows,shadows}
  \usetikzlibrary{decorations.pathmorphing}
  \usetikzlibrary{positioning,fit}
  \usetikzlibrary{calc}

\newcommand{\nospell}[1]{#1}
\setlength\headheight{24pt}
\newcommand\figcaption[1]{\Description{#1}\caption{#1}}
\raggedbottom
\tolerance=2000

\begin{document}

\title{Combining Object-Oriented Paradigm and Controlled Natural Language for Requirements Specification}
\author{Yegor Bugayenko}{}{}
\email{yegor256@gmail.com}
\affiliation{\institution{}\city{Moscow}\country{Russia}}
\ccsdesc[100]{Requirements Engineering}
\keywords{Requirements, Natural Language Processing}

\begin{abstract} Natural language is the dominant form of writing software requirements.
Its essential ambiguity causes inconsistency of requirements,
which leads to scope creep. On the other hand, formal requirements
specification notations such as Z, \nospell{Petri Nets}, SysML, and others are
difficult to understand by non-technical project stakeholders. They often
become a barrier between developers and requirements providers. The article
presents a controlled natural language that looks like English but
is a strongly typed object-oriented language compiled to
UML/XMI. Thus, it is easily understood, at the same time, by non-technical
people, programmers, and computers. Moreover, it is formally verifiable and
testable. It was designed, developed, and tested in three commercial
software projects in order to validate the assumption that object-oriented
design can be applied to requirements engineering at the level of
specifications writing. The article outlines key features of the language and summarizes
the experience obtained during its practical application. \end{abstract}
\maketitle

\section{Introduction}
\label{sec:intro}

Software Requirements Specification (SRS) should be correct, unambiguous,
complete, consistent, ranked for importance and/or stability, verifiable,
modifiable, and traceable~\citep{ieee830}. Many times, it was demonstrated that an SRS
document that lacks some of these properties has a high likelihood of
becoming a root cause of scope and schedule problems in a
project~\citep{wiegers03,ears09,chen09}.

Natural Language (NL) is the main presentation mean in industrial
requirements documents~\citep{kof10,sinha10}, mostly due to its flexibility.
Very often, contributors to requirements specifications are not experts in
requirements engineering (RE). Impreciseness and flexibility of NL, being
its advantage in informal human communications, turns into a serious drawback
in RE. Problems that appear in NL requirement documents include
ambiguity, vagueness, complexity, duplication, wordiness, and
\nospell{untestability}~\citep{ears09}.

To overcome problems associated with NL, some experts advocate the use of
other notations for the specification of requirements, such as
Z~\citep{woodcock96}, SysML, UML~\citep{uml}, Petri Nets, and others. However,
use of any of these non-textual notations often requires complex translation
of the source requirements, which can introduce further errors. Such
translation of requirements can serve to create an extra ``language
barrier'' between developers and stakeholders. There is also a training
overhead associated with the introduction of many notations~\citep{ears09}.

A desired solution to the outlined problem would be a Controlled Natural
Language (CNL), which would look like NL, but be precise,
unambiguous, and consistent. A few CNLs were introduced and developed over
the last decades. Closest to our work are Attempt to Controlled English
(ACE)~\citep{ace06}, Simplified Technical English (STE)~\citep{ste05}, Easy
Approach to Requirements Syntax (EARS)~\citep{ears09}, Boeing's Computer
Processable Language~\citep{clark05}, and \nospell{Schwitter's Processable
ENGlish}~\citep{peng}. However, there are a few critical drawbacks that exist,
to some degree, in all of these CNLs.

First, none can be mapped to objects hierarchy or a UML diagram.
They are focused more on linguistic instead of object-oriented semantic.
Some CNLs, like ACE, are based on first-order logic and can be mapped to
Lisp or other functional languages. However, as of today, object-oriented
systems dominate, especially in enterprise applications domain.

Second, they all require sequential flow of language constructs, while in
modern Agile software projects wiki pages is one of the most popular storage for
requirements documentation, instead of files or databases. Wiki pages, by
definition, are not sorted and may contain pieces of requirements
documentation that should be combined into a complete document in a random
order.

Third, most do not allow uncertainty, while requirements development
and requirements management processes need to be able to mark
requirements as uncertain (also known as ``TBD''). Requirements engineers
need to have an ability to add a requirement into scope at one iteration
and refine it later in a few iterations. This process, called ``progressive
elaboration'' of requirements~\citep{pmbok4}, must be supported by a CNL.

To overcome these drawbacks, a new language was introduced,
implemented, and tested.
Sec.~\ref{sec:Requs} of
the article presents the language and gives a simple example of its use,
specifying requirements of a sample software system.
Sec.~\ref{sec:lexical} briefly outlines lexical rules of the language.
Sec.~\ref{sec:syntax} describes the syntax using extended \nospell{Backus-Naur}
form and gives a few practical examples.
Sec.~\ref{sec:semantic} outlines
a semantic layer of the compiler that is intended to detect vast majority
of inconsistencies in a requirements document.
Sec.~\ref{sec:uml} gives an example of to-UML mapping implemented
by the compiler.
Sec.~\ref{sec:ambiguity}
presents a numeric metric that measures ambiguity in a requirements
specification and could be used as a guidance for system analysts.
Sec.~\ref{sec:results} presents empirical results obtained from three
commercial Java web projects over the last few years.
Sec.~\ref{sec:conclusion} presents conclusions and observations made by
the author.

\section{Controlled Natural Language}
\label{sec:Requs}

It is an object-oriented CNL that resembles English and, because of that,
can be read and understood by compilers, system analysts, end-users, and
business people. Fig.~\ref{fig:example} presents an example of a short
requirements definition document that describes a calculator that asks its
user for two floating point numbers and returns the result of their
division.

\begin{figure}
\begin{ffcode}
SuD includes: user as User.
Fraction is a "math calculator".
Fraction needs:
numerator as Float, and
denominator as Float.
UC1 where SuD divides two numbers:
1. The user creates Fraction (a fraction);
2. The fraction "calculates" Float
  (a quotient);
3. The user "receives results" using
  the quotient.
UC1/2 when "division by zero":
1. The user "fails" using "can't
  divide by zero".
\end{ffcode}
\figcaption{Software Requirements Specification of an example math calculator}
\label{fig:example}
\end{figure}

The example has some uncertainty expressed with plain English text sentences
in double quotes. This is an important feature that distinguishes the designed language from
other CNLs. Every class and method may be specified either formally or
by so-called ``informal'' text in double quotes. In NL requirements
specifications, such uncertainties are usually marked with TBD (``to be
determined'') placeholders~\citep{moynihan00}. Java equivalent of them are
\ff{TODO} comments. Informal texts in double quotes are also used to
make references from functional requirements to customer needs, interface
specifications, supplementary documents, etc.

\ff{SuD} (System under Development) is a top level class that encapsulates
all other objects and may not be ``created'' (similar to a singleton in OOP).
The \ff{Float} is a built-in class, as well as the \ff{Text}, the \ff{Integer}, and
a few others. Classes have methods, which are defined as use cases with a
single main flow and a few alternative flows~\citep{cockburn01}.

In the example, the use case \ff{UC1} is a method of the class \ff{SuD} and its
name is ``divides two numbers.'' Every flow step in a use case can either
\begin{inparaenum}[\itshape a\upshape)] \item call other method, or \item
create (instantiate) an object. \end{inparaenum} In the example, the step
\ff{1} of the \ff{UC1} is creating a new object \ff{fraction} of the class
\ff{Fraction}, while the step \ff{2} is calling a method ``calculates'' of
the \ff{fraction} in order to get back an object named \ff{quotient} of
the class \ff{Float}.

Every clause ends with a period. Order of clauses is not important
and this is yet another important feature of the language. In the example clause
at the tenth line, defining an alternative flow may appear before the use
case itself. The compiler will understand them correctly.

Fig.~\ref{fig:java} contains a close equivalent program in Java.

\begin{figure}
\begin{ffcode}
class User {
  Fraction createsFraction() {}
  void receivesResult(Float quotient) {}
  void fails(String message) {}
}
class Fraction {
  private final float numerator;
  private final float denumerator;
  Fraction(float num, float denum) {
    this.numerator = num;
    this.denumerator = denum;
  }
  Float quotient() throws DivisionByZero {
  }
}
class SuD {
  private final User user;
  void dividesTwoNumbers() {
    Fraction fraction =
      this.user.createsFraction();
    try {
      this.user.receivesResult(
        fraction.quotient()
      );
    } catch (DivisionByZero ex) {
      this.user.fails(
        "denominator can't be zero"
      );
    }
  }
}
\end{ffcode}
\figcaption{Java equivalent of Requs requirements document from the Fig.~\ref{fig:example}}
\label{fig:java}
\end{figure}

Requs, in its current version, doesn't have any lexical or syntax constructs
for non-functional requirements (NFRs) specification. This is a subject for
future research and implementation. The biggest expected challenge in this area is
inventing (or borrowing) a method of quantification of NFRs.

\section{Lexical Analysis}
\label{sec:lexical}

There are a few main lexical terms in Requs, which can be
defined by regular expressions:

\begin{ffcode*}{linenos=false}
<class> ::= /[A-Z][A-Za-z]+/
<word> ::= /[a-z]+/
<use-case-ID> ::= /UC[0-9\\.]+/
<flow-ID> ::= /[0-9]+/
<informal> ::= /"([^"]|$\vert$|\\")+"/
\end{ffcode*}

In Fig.~\ref{fig:example}, the entities \ff{SuD}, \ff{Fraction}, \ff{Float}, and
\ff{User} are classes. They start with capital letters and contain only
letters (both lowercase and capital). The \ff{user}, the \ff{fraction}, and
the \ff{quotient} are the variables bound to instances of classes.

Besides, there are a few reserved words, which may have special meaning in
certain places of the text, including \ff{the}, \ff{a}, \ff{includes},
\ff{requires}, etc.
A full list of them is defined below in Sec.~\ref{sec:syntax}.

\section{Syntax Analysis}
\label{sec:syntax}

A program consists of clauses (similar to statements in other
languages). Every clause ends with a period, like English sentences. There
are four types of clauses: class declaration, class construction, method
declaration, and alternative flow declaration.

\begin{ffcode*}{linenos=false}
<SRS> ::= <clause>+
<clause> ::= <class-declaration>
  |$\vert$| <class-construction>
  |$\vert$| <method-declaration>
  |$\vert$| <alternative-flow-declaration>
\end{ffcode*}

\subsection{Class Declaration}

Class declaration is a clause that declares a class as a sub-class of
another class or as a standalone class. The central part of the clause is
an \ff{is a} term:

\begin{ffcode*}{linenos=false}
<class-declaration> ::= <class>
  <is-a> <parent-class>
<parent-class> ::= <informal> |$\vert$| <class>
<is-a> ::= `is' ( `a' |$\vert$| `an' )
\end{ffcode*}

The second line in Fig.~\ref{fig:example} declares the class \ff{Fraction}
without any super-class but with an informal description
\ff{"math calculator"}.

\subsection{Class Construction}

Class construction is a clause that defines arguments of a class
constructor, if necessary. By default, a constructor doesn't need any
arguments:

\begin{ffcode*}{linenos=false}
<class-construction> ::= <class> <includes>
  `:' <slots>
<includes> ::= `includes' |$\vert$| `needs'
<slots> ::= <slot> ( `,' <slot> )*
<slot> ::= <variable> ( `as' <class> )?
<variable> ::= <word> `-s'?
\end{ffcode*}

Fig.~\ref{fig:example} defines the arguments of the constructor only for
the \ff{Fraction} class. The class \ff{User} don't have any arguments. The class
\ff{SuD} is a pre-defined class without arguments. The class \ff{Float} is a
pre-defined class with one argument.

Class without arguments in constructor normally means that it represents an
entity outside of system scope, like the \ff{User}.

Cardinality of associations between objects is configured by the \ff{-s}
prefix at the end of variable name. There are only
two possible relationships types supported: one-to-one and one-to-many.

\subsection{Method Declaration}

Method declaration is a clause that defines a method for a class:

\begin{ffcode*}{linenos=false}
<method-declaration> ::= <use-case-ID> `where'
<class> <signature> `:' <flows>
<signature> ::= ( <word>+ |$\vert$| <informal> )
  <subject>? <using>?
<subject> ::= <class> <binding>?
  |$\vert$| `the' <variable>
<binding> ::= `(' `a' <variable> `)'
<using> ::= `using' <subject>
  ( `and' <subject> )*
<flows> ::= <flow> ( `;' <flow> )+
<flow> ::= <flow-ID> `.'
  (
    `The' <variable> <signature>
    |$\vert$|
    `Fail' `since' <informal>
  )
\end{ffcode*}

For traceability, all methods have unique names across the entire
SRS starting with the \ff{UC} (use case). In other words,
methods are use cases owned by classes.

\subsection{Alternative Flow Declaration}

Alternative declaration is a clause that defines an alternative
flow of a method:

\begin{ffcode*}{linenos=false}
<alternative-flow-declaration> ::= <use-case-ID> `/' <flow-ID> `when' <informal> `:' <flows>
\end{ffcode*}

Alternative flow explains exceptional situations in one of the
flows of a method it refers to.

\section{Semantic Analysis}
\label{sec:semantic}

In the language, as in any pure object-oriented language, everything is
an object. Objects are grouped into classes. They have only methods
and no other public properties and may be bound to variables.
It is a strongly typed language.

The main goal of semantic analysis in the language is to make sure that:
\begin{inparaenum}[\itshape a\upshape)]
\item method calls match method declarations,
\item object creating have necessary constructor arguments,
\item variable bindings precede their usage,
\item methods return objects of required classes, and
\item failures are handled by alternative flows.
\end{inparaenum}
If any of these checks fail, the entire document is rejected.

\subsection{Method calls match declarations}

Every method call in every use case should either
\begin{inparaenum}[\itshape a\upshape)]
\item be informal (the 8th, 9th, and 11th lines in the example at Fig.~\ref{fig:example}),
\item formally refer to the method declared in a class, or
\item construct an instance of a class using a built-in method \ff{creates} (the 7th line in the example).
\end{inparaenum}

By such a strict linking of use case flows and class methods, a consistency
of the entire document is achieved. It is no longer possible to refer to
some functionality in a use case, which doesn't exist. A rare mistake during
the requirements development phase --- but very popular during requirements
management and maintenance --- when one part of the document is updated aside
from another part.

\subsection{Object creating have necessary constructor arguments}

Every time an object is instantiated using the \ff{creates} method,
the compiler checks that all required parameters of required classes are
supplied. Again, it is a very important validation that prevents inconsistency
during requirements maintenance and refinement. In the example, class
\ff{Fraction} requires two parameters to be instantiated. Any attempt to
create a fraction with just one parameter or two parameters of classes other
than the \ff{Float} would be rejected by the compiler.

\subsection{Variable bindings precede their usage}

A variable is declared and bound to its value by means of an article
\ff{a}. The seventh line of the example declares a variable
\ff{fraction}, which is later used with article \ff{the} (the eighth
line). If, by mistake, variable declaration is removed from the first step
of the use case or the entire step is deleted, compilation will fail.
Consistency of the document is ensured by this constraint.

\subsection{Methods return objects of required classes}

Similar to the check of parameter classes, every returned object is verified
for compliance with the class expected. The eighth line of the example
expects method ``calculates'' of the class \ff{Fraction} to return an object
of class \ff{Float}. The method is still informal and is not defined in
the requirements document, but when a requirements expert decides to make it
formal and specify its details in a new use case, she has to make sure it
returns an instance of the class \ff{Float}. Otherwise, the compiler would
complain with an error.

\subsection{Failures are handled by alternative flows}

Every method in the object-oriented way may throw an exception, using a
build-in method \ff{Fail since}. The method has one parameter of the class
\ff{Text}, which explains the reason. In the example in
Fig.~\ref{fig:example}, method ``calculates'' of the class \ff{Fraction} may
throw an exception and it is handled by an alternative flow. In the language, all
exceptions are checked, which means that they should be handled explicitly,
by means of alternative flows of use cases.

\section{UML and XMI}
\label{sec:uml}
\definecolor{body}{rgb}{1,1,0.8}
\definecolor{border}{rgb}{0.6039,0,0.2}
\tikzstyle{uml} = [align=left,fill=body,draw=border,font={\small\ttfamily}]
\tikzstyle{uml-class} = [uml,text ragged,rectangle split,
rectangle split parts=3,rectangle split empty part height=0em,
rectangle split every empty part={}]
\tikzstyle{comment} = [inner sep=0mm,fill=white,font={\scriptsize\ttfamily}]

When semantic analysis is done the specification looks like an object of
the class \ff{SuD} that encapsulates other objects and methods. This object
hierarchy is convertible to UML diagrams~\citep{uml}. Specification in
Fig.~\ref{fig:example} would be converted to the following UML diagrams
(to save space, the list is limited to the three most interesting
diagrams):

\begin{itemize}
\item Class diagram, in Fig.~\ref{fig:uml-class}
\item Object diagram, in Fig.~\ref{fig:uml-object}
\item Sequence diagram of ``divides two numbers'' method, in Fig.~\ref{fig:uml-seq}
\end{itemize}

\begin{figure}
\begin{tikzpicture}[node distance=1em]
\node[uml-class] (sud) {SuD\nodepart{third}divides two numbers};
\node[uml-class,below right=of sud] {Float\nodepart{third}};
\node[uml-class,below left=of sud] {User\nodepart{third}creates\\receives results\\fails};
\node[uml-class,below=of sud] {Fraction\nodepart{third}calculates};
\end{tikzpicture}
\figcaption{UML class diagram for the sample project}
\label{fig:uml-class}
\end{figure}

\begin{figure}
\begin{tikzpicture}[node distance=2.5em]
\node[uml] (sud) {\underline{:SuD}};
\node[uml,below right=of sud] (user) {\underline{:User}};
\node[uml,below left=6em of user] (fraction) {\underline{:Fraction}};
\node[uml,below left=of fraction] (numerator) {\underline{:Float}};
\node[uml,below right=of fraction] (denumerator) {\underline{:Float}};
\node[uml,left=9em of user] (quotient) {\underline{:Float}};
\path[draw,diamond-] (sud) -- node[midway,comment] {user} (user);
\path[draw,diamond-] (fraction) -- node[midway,comment] {numerator} (numerator);
\path[draw,diamond-] (fraction) -- node[midway,comment] {denumerator} (denumerator);
\path[draw,->,dashed] (user) -- node[midway,comment] {creates} (fraction);
\path[draw,->,dashed] (fraction) -- node[midway,comment] {calculates} (quotient);
\path[draw,->,dashed] (quotient) -- node[midway,comment] {receives results} (user);
\end{tikzpicture}
\figcaption{UML object diagram for the sample project}
\label{fig:uml-object}
\end{figure}

\begin{figure}
\begin{sequencediagram}
\tikzstyle{inststyle}=[uml,rectangle,draw,anchor=west,minimum width=1.6cm,align=center]
\newthread{user}{user}
\newinst[1]{fraction}{fraction}
\newinst[1]{quotient}{quotient}
\begin{messcall}{user}{create}{fraction}
\begin{sdblock}{try/catch}{}
\begin{call}{fraction}{calculates}{quotient}{quotient}
\end{call}
\begin{messcall}{fraction}{receives result}{user}
\end{messcall}
\begin{callself}{fraction}{division by zero}{}
\end{callself}
\begin{messcall}{fraction}{receive}{user}
\end{messcall}
\end{sdblock}
\path [draw] (0.5,-4.7) -- (7,-4.7);
\end{messcall}
\end{sequencediagram}
\figcaption{UML sequence diagram for ``divides two numbers'' method of class SuD of the sample project}
\label{fig:uml-seq}
\end{figure}

Every UML diagram is a formal interoperable document in XMI~\citep{xmi} that
can be rendered in web, in \LaTeX, in PDF, or translated to Java or any
other object-oriented language. Fig.~\ref{fig:xmi} shows an example XMI
for the class diagram.

\begin{figure}
\begin{ffcode}
<uml:Model xmi:version="2.1"
  xmlns:uml
    ="http://www.eclipse.org/uml2/2.0.0/UML"
  xmlns:xmi
    ="http://schema.omg.org/spec/XMI/2.1">
  <packagedElement name="SuD"
    xmi:type="uml:Class">
    <ownedOperation name="divides two numbers"
      xmi:type="uml:Operation">
      <ownedParameter direction="return"
        xmi:type="uml:Parameter"/>
    </ownedOperation>
  </packagedElement>
  <packagedElement name="User"
    xmi:type="uml:Class">
    <ownedOperation name="creates fraction"
      xmi:type="uml:Operation">
      <ownedParameter direction="return"
        xmi:type="uml:Parameter"/>
    </ownedOperation>
  </packagedElement>
  [...]
</uml:Model>
\end{ffcode}
\figcaption{XMI representation of a UML diagram; a sample part}
\label{fig:xmi}
\end{figure}

\section{Ambiguity}
\label{sec:ambiguity}

With this approach, ambiguity of requirements becomes a formally measurable metric.
It is calculated as a division of informally defined methods to the total
number of methods:

\begin{equation}
A = \frac{M_{\mbox{informal}}}{M_{\mbox{total}}},\quad 0 \le A \le 1
\end{equation}

In Fig.~\ref{fig:example}, ambiguity equals to $3/4$ since there are four
methods in total and three of them are defined informally.

A method is formally defined when its signature consists of one specially
reserved verb \ff{creates} and an optional list of constructor arguments,
as in the seventh line of Fig.~\ref{fig:example}. A formally defined
method doesn't have any ambiguity since it is absolutely clear what is
expected as its input and output. All it does is instantiate a new object
and bind it to a variable. In Java that would mean calling a constructor.

The higher the ambiguity, the more system analysis is required for the
requirements document to make it unambiguously understandable by everybody
in a project. The ultimate goal of a system analyst is to break down
requirements until a target ambiguity is reached. In the commercial projects
explained in Sec.~\ref{sec:results}, the $A$ metric was used for the planning
of RE activities. Every iteration a group of RE specialists had a goal of
decreasing the ambiguity until it reached a given value. It was not clear at
the beginning how much work that would take, but soon the RE team
obtained that knowledge and became capable of estimating
complexity of work using $A$ metric.

For example, in one of the projects, a decrease of ambiguity from $0.86$ to
$0.8$ took five working hours of a system analyst. During this work, eight
tickets were produced for discussion, each of which took from one to three
hours of work of requirements providers and other project stakeholders.
Thus, a total estimate of 0.06 decrease of ambiguity costs approximately
twenty hours of work. This is a very rough estimate and is applicable only
to the project where it was measured, but in other projects it is possible
to calculate similar metrics and manage RE activities according to them.

It was empirically observed that an SRS document becomes acceptable for
implementation and doesn't confuse programmers when its ambiguity is less
than $0.7$. Of course, it is recommended to start implementation at earlier
stages of a project, when ambiguity is rather higher, and generate requests
for the RE team to improve SRS at certain places, where ambiguity has
its peaks.

\section{Empirical Results and Future Work}
\label{sec:results}

The language was used in three commercial projects. The numbers
empirically collected are presented in Table~\ref{table:numbers}.

\begin{table}
\begin{tabular}{>{\raggedright}p{14em}rrr}
                                            & $p_1$ & $p_2$ & $p_3$ \\
\hline
Total classes                                 & 39 & 12 & 19 \\
Total methods                                 & 19 & 7 & 24 \\
Total Java classes                                  & 80 & 75 & 295 \\
Ambiguity achieved                                  & 0.80 & 0.95 & 0.68 \\
Work spent on RE, staff-months                      & 0.53 & 0.21 & 0.86 \\
Total SRS contributors                              & 4 & 7 & 2 \\
Non-empty lines of Java code                        & 9.5K & 10K & 32K \\
Project duration, months                            & 7 & 26 & 11 \\
Project budget, staff-months                        & 5.5 & 7.5 & 9.5 \\
\hline
\end{tabular}
\figcaption{Empirically collected data from three commercial projects}
\label{table:numbers}
\end{table}

The results collected from three software projects demonstrated that
combining object-oriented paradigm with a CNL makes it possible to
specify software requirements in a predictable and verifiable manner. The
quality of requirements documentation was higher than in previously
developed projects, according to the estimates of defects reported during
project development and release phases.

It seems reasonable to analyze the effectiveness of the created CNL on
a larger number of projects, including open source ones. In order to make
such an analysis a methodology would have to be designed, to compare
the effectiveness of different methods of requirements specification
and identifying the advantages and disadvantages of each one.

\section{Conclusions}
\label{sec:conclusion}

A few important observations were made during commercial usage of the language.
First, SRS document is much shorter than discussions around it (usually kept
in bug tracking ``tickets''). It takes days to discuss one small change in a
class and just a few minutes to apply the change. Because of that, a
traceability from SRS to discussion tickets is an important feature that
has to be supported by a documentation management software. It is crucial
to have an ability to trace back every requirement and remember
the reasons behind it.

Second, complete and detailed error reporting is an important aspect of
the compiler. Initial versions of it had a technical and simplistic
errors reporting mechanism that didn't give enough information to system
analysts and business owners. They were confused trying to edit a part of
SRS in one of wiki pages and receiving a message like ``incorrect syntax on
line 45.'' Such situations required immediate attention from
programmers. Soon it became obvious that the compiler, unlike
programmers-oriented compilers of Java or C++, has to produce user friendly
error messages (and warnings). In further versions of the language, a much richer
reporting is going to be implemented.

Third, developing requirements specification with the language requires RE
engineers to understand key principles of ``object thinking''~\citep{west04},
like ``everything is an object'' and ``objects expose behavior, not state.''
Initial training was required in two projects.

\bibliographystyle{ACM-Reference-Format}
\raggedright
\bibliography{main}
\clearpage

\end{document}
